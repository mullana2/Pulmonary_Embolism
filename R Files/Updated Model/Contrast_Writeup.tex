\documentclass[]{article}
\usepackage{lmodern}
\usepackage{amssymb,amsmath}
\usepackage{ifxetex,ifluatex}
\usepackage{fixltx2e} % provides \textsubscript
\ifnum 0\ifxetex 1\fi\ifluatex 1\fi=0 % if pdftex
  \usepackage[T1]{fontenc}
  \usepackage[utf8]{inputenc}
\else % if luatex or xelatex
  \ifxetex
    \usepackage{mathspec}
  \else
    \usepackage{fontspec}
  \fi
  \defaultfontfeatures{Ligatures=TeX,Scale=MatchLowercase}
\fi
% use upquote if available, for straight quotes in verbatim environments
\IfFileExists{upquote.sty}{\usepackage{upquote}}{}
% use microtype if available
\IfFileExists{microtype.sty}{%
\usepackage{microtype}
\UseMicrotypeSet[protrusion]{basicmath} % disable protrusion for tt fonts
}{}
\usepackage[margin=1in]{geometry}
\usepackage{hyperref}
\hypersetup{unicode=true,
            pdftitle={Contrast\_Writeup},
            pdfauthor={Aidan Mullan},
            pdfborder={0 0 0},
            breaklinks=true}
\urlstyle{same}  % don't use monospace font for urls
\usepackage{graphicx,grffile}
\makeatletter
\def\maxwidth{\ifdim\Gin@nat@width>\linewidth\linewidth\else\Gin@nat@width\fi}
\def\maxheight{\ifdim\Gin@nat@height>\textheight\textheight\else\Gin@nat@height\fi}
\makeatother
% Scale images if necessary, so that they will not overflow the page
% margins by default, and it is still possible to overwrite the defaults
% using explicit options in \includegraphics[width, height, ...]{}
\setkeys{Gin}{width=\maxwidth,height=\maxheight,keepaspectratio}
\IfFileExists{parskip.sty}{%
\usepackage{parskip}
}{% else
\setlength{\parindent}{0pt}
\setlength{\parskip}{6pt plus 2pt minus 1pt}
}
\setlength{\emergencystretch}{3em}  % prevent overfull lines
\providecommand{\tightlist}{%
  \setlength{\itemsep}{0pt}\setlength{\parskip}{0pt}}
\setcounter{secnumdepth}{0}
% Redefines (sub)paragraphs to behave more like sections
\ifx\paragraph\undefined\else
\let\oldparagraph\paragraph
\renewcommand{\paragraph}[1]{\oldparagraph{#1}\mbox{}}
\fi
\ifx\subparagraph\undefined\else
\let\oldsubparagraph\subparagraph
\renewcommand{\subparagraph}[1]{\oldsubparagraph{#1}\mbox{}}
\fi

%%% Use protect on footnotes to avoid problems with footnotes in titles
\let\rmarkdownfootnote\footnote%
\def\footnote{\protect\rmarkdownfootnote}

%%% Change title format to be more compact
\usepackage{titling}

% Create subtitle command for use in maketitle
\newcommand{\subtitle}[1]{
  \posttitle{
    \begin{center}\large#1\end{center}
    }
}

\setlength{\droptitle}{-2em}
  \title{Contrast\_Writeup}
  \pretitle{\vspace{\droptitle}\centering\huge}
  \posttitle{\par}
  \author{Aidan Mullan}
  \preauthor{\centering\large\emph}
  \postauthor{\par}
  \predate{\centering\large\emph}
  \postdate{\par}
  \date{3/7/2019}

\usepackage{float}

\begin{document}
\maketitle

\subsection{Contrast Distributions}\label{contrast-distributions}

Using the original dataset to start, all patients that were recorded to
have received more than 150cc of contrast were removed. This left us
with 2328 patients total. Figure 1 shows the distribution of mean
contrast administered by age group and BMI group. Here age is grouped by
10-year intervals and BMI is grouped at 5-unit intervals.

\includegraphics{Contrast_Writeup_files/figure-latex/unnamed-chunk-1-1.pdf}

Here we see a somewhat quadratic relationship between Age and contrast,
which is similar to what we saw for the distribution of radiation
administered. We also have a linear association between BMI and
contrast, with the amount of contrast being administered to a patient
increasing as the patient's BMI increases.

One question that arises is why do we see a quadratic relationship
between Age and Contrast? It may be due to some quadratic relationship
between age and BMI, where young and old patients tend to weigh less
than patients between 30 and 60. Figure 2 overlays the relationship
between age and BMI onto the relationship between age and contrast so
that we can compare.

In this graph we do see similar relationships with age for BMI and
contrast, which seems to support our theory that there is a quadratic
relationship between patient age and BMI that can explain the quadratic
nature of the relationship between patient age and the amound of
contrast administered prior to the CT scan.

\includegraphics{Contrast_Writeup_files/figure-latex/unnamed-chunk-2-1.pdf}

\subsection{Contrast Model}\label{contrast-model}

Then, a linear model was fit for contrast considering age
(\textless{}18, 18-35, \textgreater{}35), BMI (\textless{}20, 20-40,
\textgreater{}40), Gender, Location of Admission, and Type of CT scanner
used. Model coefficients are given in Table 1.

\begin{table}[H] \centering 
  \caption{Positivity Model Coefficients} 
\begin{tabular}{p{3cm}|p{3cm}p{3cm}p{3cm}}
\\[-1.8ex] \hline 
\hline
 & Estimate & Std. Error & p  \\
\hline \\[-1.8ex] 
 Intercept   & 67.332 & 3.490 & <.001 $***$ \\
 BMI 25-40   & 2.847 & 0.742 & <.001 $***$\\
 BMI $>$40   & 16.443 & 1.004 & <.001 $***$\\
 Loc: ICU    & 2.480 & 1.046 & .018 $*$\\
 Loc: IN     & 0.821 & 0.807 & .309 \\
 Loc: OUT    & 2.988 & 1.292 & .021 $.$\\
 CT: 64SSwIR & 4.771 & 1.647 & .004 $**$\\
 CT: 64SSWoIR & 5.752 & 2.143 & .004 $**$ \\
 CT: DS Scanner & $-$9.316 & 0.676 & <.001 $***$\\
 Age: 18-35 & 19.476 & 3.521 & <.001 $***$\\
 Age: $>$35 & 18.577 & 3.411 & <.001 $***$ \\
 Gender: Male & 4.162 & 0.648 & <.001 $***$ \\
\hline 
\hline \\[-1.8ex]
\multicolumn{4}{l}{\textit{Note:} Significance: $.$ < .1, $*$ <.05, $**$ < .01, $***$ < .001}
 \end{tabular}
\end{table}

\paragraph{Influence of Scanner Type}\label{influence-of-scanner-type}

Our contrast model found a significant effect for the type of scanner
used. Post-hoc comparisons conducted using Tukey's HSD found that the DS
scanners were associated with significantly lower levels of contrast
than all 3 other scanners (\(p<.001\) for all 3). Additionally, the
64SSwIR and 64SSwoIR scanners had significantly higher levels of
contrast than the 128SSwIR scanner (64SSwIR: \(p = .001\); 64SSwoIR:
\(p = .016\)). There was no significant difference in contrast usage
between the 64SSwIR and 64SSwoIR scanners.

From the model, these significant differences can be interpreted as:

\begin{itemize}
\item On average, patients who receive a CT using the DS scanner are administered 9.32cc less contrast than patients on the 128SSwIR scanner, 14.09cc less contrast than patients on the 64SSwIR scanner, and 15.07cc less contrast than patients on the 64SSwoIR scanner
\item On average, patients who receive a CT using the 128SSwIR scanner are administered 4.77cc less contrast than patients on the 64SSwIR scanner and 5.75cc less contrast than patients on the 64SSwoIR scanner.
\end{itemize}

\subsection{Radiation and Age}\label{radiation-and-age}

In earlier write-ups, we have seen a quadratic relationship between
radiation administered and patient age. There appears to be a peak in
effective dosage that occurs somewhere between 30 and 60 years of age.
To better pinpoint exactly where this peak occurs, we can look at the
average effective dose administered to patients grouped by age in 1-year
intervals (Figure 3a). Here we still see the quadratic relationship, but
there is a substantial amount of noise in the graph, which makes it
rather difficult to identify the exact year at which effective dose
reaches a maximum.

To better determine the age at which the maximum radiation is
administered, we can smooth this curve to get a better sense of the
general pattern which will isolate the maximum effective dose. This
smoothing process can be seen in Figures 3b. From this we get that the
maximum average effective dose was administered to patients age 46.
However, we can see that a more general range for the maximum radiation
is for patients between 43 and 49.

\includegraphics{Contrast_Writeup_files/figure-latex/unnamed-chunk-3-1.pdf}

\newpage

\subsection{Patient Population}\label{patient-population}

Several descriptive statistics for the general patient population were
requested for the papers that will be written about this data. I have
included this information here.

For the total patient population (Chest Only and CAP), there were 2713
patients. This breaks down into 1278 male (47.1\%) and 1435 (52.9\%)
female patients. For the chest only condition, there were a total of
2328 patients. Out of these, 1093 (47.0\%) were male and 1235 (53./\%)
were female.

For the chest only patients, the range of DLP was 7.0 to 3072.0, with a
mean of 393.7 and a median of 305.0.

\subsection{Manuscript Results
Section}\label{manuscript-results-section}

This section is a direct re-write of the manuscript that I was sent for
the effective dosage and contrast models, with the missing statistics
filled in.

\paragraph{Effective Dose Model}\label{effective-dose-model}

There were 2369 total patients (mean age 58.1 years; age range 0.2-104.4
years) with 1099 male patients (mean age 59.42; age range 0.2-96.5
years) and 1243 female patients (mean age 56.9; age range 0.6-104.4
years). The mean effective radiation dose was 5.512 mSv (median dose
4.27 mSv; dose range 0.1-43.0 mSv). A linear regression model was used
to determine which factors were significant indicators for the amount of
radiation administered to a patient. Since the distribution of effective
dose was right skewed, the linear model was fit to the log-transform of
effective dose.

The model found significant with patient's age group (\textless{}18
years, 18-35 years, \textgreater{}35 years), BMI group (BMI under 20,
between 20-40, above 40), gender, location of patient admittance, and
type of scanner used for CTA. All post-hoc pairwise comparisons were
then performed using Tukey's HSD with a FDR correction to compensate for
the large number of factor levels being compared.

Mean radiation dose for the under 18 age group was 3.18 mSv, which was
significantly lower than the average dose for the 18-35 group (mean:
5.24 mSv, p \textless{} .001) and over 35 group (mean: 5.57 mSv, p
\textless{} .001). There was no significant difference between the 18-35
and over 35 groups. Patient body habitus was a strong predictor for
radiation dose with a mean dose of 3.21 mSv for the normal weight group
{[}BMI \textless{}25{]}, which was significantly lower than the
overweight group {[}BMI 25-40{]} (mean: 5.28 mSv, p \textless{} .001)
and morbidly obese group {[}BMI \textgreater{}40{]} (mean: 10.25 mSv, p
\textless{} .001). The mean radiation dose administered to the morbidly
obese group was also significantly higher than the overweight group (p
\textless{} .001).

For the location at which the patient was admitted, intensive care unit
(ICU) patients were associated with the highest radiation dose at an
average of 6.16 mSv, compared to an average of 5.43 mSv for all other
locations (p \textless{} .001). For the type of scanner used, the DS
scanners had an average effective dose of 4.90 mSv, which was
significantly lower than the 64SSwIR scanner (mean: 6.08 mSv, p
\textless{} .001), 128SSwIR scanner (mean: 5.94 mSv, p \textless{}
.001), and 64SSwoIR scanner (mean: 9.29 mSv, p \textless{} .001).
Additionally, the 64SSwoIR scanner was associated with a significantly
higher amount of radiation than all other scanners (p \textless{} .001
for all). There was no difference in administered radiation between the
64SSwIR and 128SSwIR scanners.

\newpage

\paragraph{Contrast Model}\label{contrast-model-1}

Analysis of the amount of contrast administered to patients used the
same patient base as in the effective dose study. The mean contrast dose
was 88.48 mL (median dose 95.0 mL; dose range 5.0-150.0 mL). A linear
regression model was used to determine which factors were significant
indicators for the amount of radiation administered to a patient.

The model found significant with patient's age group (\textless{}18
years, 18-35 years, \textgreater{}35 years), BMI group (BMI under 20,
between 20-40, above 40), gender, location of patient admittance, and
type of scanner used for CTA. All post-hoc pairwise comparisons were
then performed using Tukey's HSD with a FDR correction to compensate for
the large number of factor levels being compared.

Mean contrast dose for the under 18 age group was 65.67 mL, which was
significantly lower than the average dose for the 18-35 group (mean:
88.29 mL, p \textless{} .001) and over 35 group (mean: 88.74 mL, p
\textless{} .001). There was no significant difference between the 18-35
and over 35 groups. Patient body habitus was a strong predictor for the
amount of contrast administered with a mean dose of 84.46 mL for the
normal weight group {[}BMI \textless{}25{]}, which was significantly
lower than the overweight group {[}BMI 25-40{]} (mean: 87.23 mL, p
\textless{} .001) and morbidly obese group {[}BMI \textgreater{}40{]}
(mean: 99.68 mL, p \textless{} .001). The mean radiation dose
administered to the morbidly obese group was also significantly higher
than the overweight group (p \textless{} .001).

For the location at which the patient was admitted, intensive care unit
(ICU) patients were associated with the highest contrast dose at an
average of 90.58 mL, compared to an average of 88.21 mL for all other
locations. Although this effect was found to be significant in the
model, the difference was found to be only non-significant during
post-hoc analysis with the FDR correction (p = .171). For the type of
scanner used, the DS scanners had an average contrast dose of 83.45 mL,
which was significantly lower than the 128SSwIR scanner (mean: 92.67 mL,
p \textless{} .001), 64SSwIR scanner (mean: 98.47 mL, p \textless{}
.001), and 64SSwoIR scanner (mean: 98.84 mL mSv, p \textless{} .001).
Additionally, the 128SSwIR scanner was associated with a significantly
lower amount of contrast than the 64SSwIR (p = .006). Only marginal
significance was found for the difference between the 128SSwIR and
64SSwoIR scanners (p = .064), and no difference was found between the
64SSwIR and 64SSwoIR scanners.

\paragraph{Other Details}\label{other-details}

All analysis was conducted using R statistical software, version 3.5.0.


\end{document}
