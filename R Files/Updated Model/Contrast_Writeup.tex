\documentclass[]{article}
\usepackage{lmodern}
\usepackage{amssymb,amsmath}
\usepackage{ifxetex,ifluatex}
\usepackage{fixltx2e} % provides \textsubscript
\ifnum 0\ifxetex 1\fi\ifluatex 1\fi=0 % if pdftex
  \usepackage[T1]{fontenc}
  \usepackage[utf8]{inputenc}
\else % if luatex or xelatex
  \ifxetex
    \usepackage{mathspec}
  \else
    \usepackage{fontspec}
  \fi
  \defaultfontfeatures{Ligatures=TeX,Scale=MatchLowercase}
\fi
% use upquote if available, for straight quotes in verbatim environments
\IfFileExists{upquote.sty}{\usepackage{upquote}}{}
% use microtype if available
\IfFileExists{microtype.sty}{%
\usepackage{microtype}
\UseMicrotypeSet[protrusion]{basicmath} % disable protrusion for tt fonts
}{}
\usepackage[margin=1in]{geometry}
\usepackage{hyperref}
\hypersetup{unicode=true,
            pdftitle={Contrast\_Writeup},
            pdfauthor={Aidan Mullan},
            pdfborder={0 0 0},
            breaklinks=true}
\urlstyle{same}  % don't use monospace font for urls
\usepackage{graphicx,grffile}
\makeatletter
\def\maxwidth{\ifdim\Gin@nat@width>\linewidth\linewidth\else\Gin@nat@width\fi}
\def\maxheight{\ifdim\Gin@nat@height>\textheight\textheight\else\Gin@nat@height\fi}
\makeatother
% Scale images if necessary, so that they will not overflow the page
% margins by default, and it is still possible to overwrite the defaults
% using explicit options in \includegraphics[width, height, ...]{}
\setkeys{Gin}{width=\maxwidth,height=\maxheight,keepaspectratio}
\IfFileExists{parskip.sty}{%
\usepackage{parskip}
}{% else
\setlength{\parindent}{0pt}
\setlength{\parskip}{6pt plus 2pt minus 1pt}
}
\setlength{\emergencystretch}{3em}  % prevent overfull lines
\providecommand{\tightlist}{%
  \setlength{\itemsep}{0pt}\setlength{\parskip}{0pt}}
\setcounter{secnumdepth}{0}
% Redefines (sub)paragraphs to behave more like sections
\ifx\paragraph\undefined\else
\let\oldparagraph\paragraph
\renewcommand{\paragraph}[1]{\oldparagraph{#1}\mbox{}}
\fi
\ifx\subparagraph\undefined\else
\let\oldsubparagraph\subparagraph
\renewcommand{\subparagraph}[1]{\oldsubparagraph{#1}\mbox{}}
\fi

%%% Use protect on footnotes to avoid problems with footnotes in titles
\let\rmarkdownfootnote\footnote%
\def\footnote{\protect\rmarkdownfootnote}

%%% Change title format to be more compact
\usepackage{titling}

% Create subtitle command for use in maketitle
\newcommand{\subtitle}[1]{
  \posttitle{
    \begin{center}\large#1\end{center}
    }
}

\setlength{\droptitle}{-2em}
  \title{Contrast\_Writeup}
  \pretitle{\vspace{\droptitle}\centering\huge}
  \posttitle{\par}
  \author{Aidan Mullan}
  \preauthor{\centering\large\emph}
  \postauthor{\par}
  \predate{\centering\large\emph}
  \postdate{\par}
  \date{3/7/2019}

\usepackage{float}

\begin{document}
\maketitle

\subsection{Contrast Distributions}\label{contrast-distributions}

Using the original dataset to start, all patients that were recorded to
have received more than 150cc of contrast were removed. This left us
with 2365 patients total. Figure 1 shows the distribution of mean
contrast administered by age group and BMI group. Here age is grouped by
10-year intervals and BMI is grouped at 5-unit intervals.

\includegraphics{Contrast_Writeup_files/figure-latex/unnamed-chunk-1-1.pdf}

Here we see a somewhat quadratic relationship between Age and contrast,
which is similar to what we saw for the distribution of radiation
administered. We also have a linear association between BMI and
contrast, with the amount of contrast being administered to a patient
increasing as the patient's BMI increases.

One question that arises is why do we see a quadratic relationship
between Age and Contrast? It may be due to some quadratic relationship
between age and BMI, where young and old patients tend to weigh less
than patients between 30 and 60. Figure 2 overlays the relationship
between age and BMI onto the relationship between age and contrast so
that we can compare.

In this graph we do see similar relationships with age for BMI and
contrast, which seems to support our theory that there is a quadratic
relationship between patient age and BMI that can explain the quadratic
nature of the relationship between patient age and the amound of
contrast administered prior to the CT scan.

\includegraphics{Contrast_Writeup_files/figure-latex/unnamed-chunk-2-1.pdf}

\subsection{Contrast Model}\label{contrast-model}

Then, a linear model was fit for contrast considering age
(\textless{}18, 18-35, \textgreater{}35), BMI (\textless{}20, 20-40,
\textgreater{}40), Gender, Location of Admission, and Type of CT scanner
used. Model coefficients are given in Table 1.

\begin{table}[H] \centering 
  \caption{Positivity Model Coefficients} 
\begin{tabular}{p{3cm}|p{3cm}p{3cm}p{3cm}}
\\[-1.8ex] \hline 
\hline
 & Estimate & Std. Error & p  \\
\hline \\[-1.8ex] 
 Intercept   & 67.332 & 3.490 & <.001 $***$ \\
 BMI 25-40   & 2.847 & 0.742 & <.001 $***$\\
 BMI $>$40   & 16.443 & 1.004 & <.001 $***$\\
 Loc: ICU    & 2.480 & 1.046 & .018 $*$\\
 Loc: IN     & 0.821 & 0.807 & .309 \\
 Loc: OUT    & 2.988 & 1.292 & .021 $.$\\
 CT: 64SSwIR & 4.771 & 1.647 & .004 $**$\\
 CT: 64SSWoIR & 5.752 & 2.143 & .004 $**$ \\
 CT: DS Scanner & $-$9.316 & 0.676 & <.001 $***$\\
 Age: 18-35 & 19.476 & 3.521 & <.001 $***$\\
 Age: $>$35 & 18.577 & 3.411 & <.001 $***$ \\
 Gender: Male & 4.162 & 0.648 & <.001 $***$ \\
\hline 
\hline \\[-1.8ex]
\multicolumn{4}{l}{\textit{Note:} Significance: $.$ < .1, $*$ <.05, $**$ < .01, $***$ < .001}
 \end{tabular}
\end{table}

\paragraph{Influence of Scanner Type}\label{influence-of-scanner-type}

Our contrast model found a significant effect for the type of scanner
used. Post-hoc comparisons conducted using Tukey's HSD found that the DS
scanners were associated with significantly lower levels of contrast
than all 3 other scanners (\(p<.001\) for all 3). Additionally, the
64SSwIR and 64SSwoIR scanners had significantly higher levels of
contrast than the 128SSwIR scanner (64SSwIR: \(p = .001\); 64SSwoIR:
\(p = .016\)). There was no significant difference in contrast usage
between the 64SSwIR and 64SSwoIR scanners.

From the model, these significant differences can be interpreted as:

\begin{itemize}
\item On average, patients who receive a CT using the DS scanner are administered 9.32cc less contrast than patients on the 128SSwIR scanner, 14.09cc less contrast than patients on the 64SSwIR scanner, and 15.07cc less contrast than patients on the 64SSwoIR scanner
\item On average, patients who receive a CT using the 128SSwIR scanner are administered 4.77cc less contrast than patients on the 64SSwIR scanner and 5.75cc less contrast than patients on the 64SSwoIR scanner.
\end{itemize}


\end{document}
