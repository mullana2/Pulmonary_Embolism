\documentclass[]{article}
\usepackage{lmodern}
\usepackage{amssymb,amsmath}
\usepackage{ifxetex,ifluatex}
\usepackage{fixltx2e} % provides \textsubscript
\ifnum 0\ifxetex 1\fi\ifluatex 1\fi=0 % if pdftex
  \usepackage[T1]{fontenc}
  \usepackage[utf8]{inputenc}
\else % if luatex or xelatex
  \ifxetex
    \usepackage{mathspec}
  \else
    \usepackage{fontspec}
  \fi
  \defaultfontfeatures{Ligatures=TeX,Scale=MatchLowercase}
\fi
% use upquote if available, for straight quotes in verbatim environments
\IfFileExists{upquote.sty}{\usepackage{upquote}}{}
% use microtype if available
\IfFileExists{microtype.sty}{%
\usepackage{microtype}
\UseMicrotypeSet[protrusion]{basicmath} % disable protrusion for tt fonts
}{}
\usepackage[margin=1in]{geometry}
\usepackage{hyperref}
\hypersetup{unicode=true,
            pdftitle={Update\_Writeup},
            pdfauthor={Aidan Mullan},
            pdfborder={0 0 0},
            breaklinks=true}
\urlstyle{same}  % don't use monospace font for urls
\usepackage{graphicx,grffile}
\makeatletter
\def\maxwidth{\ifdim\Gin@nat@width>\linewidth\linewidth\else\Gin@nat@width\fi}
\def\maxheight{\ifdim\Gin@nat@height>\textheight\textheight\else\Gin@nat@height\fi}
\makeatother
% Scale images if necessary, so that they will not overflow the page
% margins by default, and it is still possible to overwrite the defaults
% using explicit options in \includegraphics[width, height, ...]{}
\setkeys{Gin}{width=\maxwidth,height=\maxheight,keepaspectratio}
\IfFileExists{parskip.sty}{%
\usepackage{parskip}
}{% else
\setlength{\parindent}{0pt}
\setlength{\parskip}{6pt plus 2pt minus 1pt}
}
\setlength{\emergencystretch}{3em}  % prevent overfull lines
\providecommand{\tightlist}{%
  \setlength{\itemsep}{0pt}\setlength{\parskip}{0pt}}
\setcounter{secnumdepth}{0}
% Redefines (sub)paragraphs to behave more like sections
\ifx\paragraph\undefined\else
\let\oldparagraph\paragraph
\renewcommand{\paragraph}[1]{\oldparagraph{#1}\mbox{}}
\fi
\ifx\subparagraph\undefined\else
\let\oldsubparagraph\subparagraph
\renewcommand{\subparagraph}[1]{\oldsubparagraph{#1}\mbox{}}
\fi

%%% Use protect on footnotes to avoid problems with footnotes in titles
\let\rmarkdownfootnote\footnote%
\def\footnote{\protect\rmarkdownfootnote}

%%% Change title format to be more compact
\usepackage{titling}

% Create subtitle command for use in maketitle
\newcommand{\subtitle}[1]{
  \posttitle{
    \begin{center}\large#1\end{center}
    }
}

\setlength{\droptitle}{-2em}
  \title{Update\_Writeup}
  \pretitle{\vspace{\droptitle}\centering\huge}
  \posttitle{\par}
  \author{Aidan Mullan}
  \preauthor{\centering\large\emph}
  \postauthor{\par}
  \predate{\centering\large\emph}
  \postdate{\par}
  \date{2/26/2019}

\usepackage{float}

\begin{document}
\maketitle

\subsection{Effective Dose and
Contrast}\label{effective-dose-and-contrast}

\paragraph{Summary Statistics}\label{summary-statistics}

Numerical summaries of both effective dose and contrast are given in
Table 1. Using the fixed data for effective dosage of radiation
administered to patients, we retrieve the plots shown in Figure 1. Here
we can see the right-skew in the histogram suggesting several patients
recieved much higher amounts of radiation than average.

\begin{table}[H] \centering 
  \caption{Summary Statistics of Radiation and Contrast} 
\begin{tabular}{p{3cm}p{1cm}p{1cm}p{1cm}p{1cm}p{1cm}p{1cm}}
\\[-1.8ex] \hline 
\hline
 & Min & 1st Q & Median & Mean & 3rd Q & Max \\ 
\hline \\[-1.8ex] 
Effective Dose & 0.10 & 3.06 & 4.30 & 5.631 & 6.36 & 48.83 \\
Contrast       & 5.0 & 75.0 & 95.0 & 88.5 & 98.0 & 150.0 \\
\hline 
\hline \\[-1.8ex]
 \end{tabular}
\end{table}

\includegraphics{Updates_Writeup_files/figure-latex/unnamed-chunk-2-1.pdf}

We also have constructed an updated histogram and boxplot for the amount
of contrast given to patients, which are shown in Figure 2. Here we see
a centralized distribution with some extreme values on both sides.

\includegraphics{Updates_Writeup_files/figure-latex/unnamed-chunk-3-1.pdf}

We are also interested in the relationship of the effecive dose and
contrast administered with both the age and BMI of the patient. Patients
were divided into 10-year age groups and 5-unit BMI groups. In both
cases, cut-off values were determined based on the number of patients
that fell into the extreme groups. For example, there was only one
patient in the data under the age of 10, so they were added into the
``\textless{}20'' group. The number of patients that belong to each age
group and BMI group is given in Table 2 and 3 respectively.

\begin{table}[H] \centering 
  \caption{Breakdown Patients by Age Group} 
\begin{tabular}{p{1cm}p{1cm}p{1cm}p{1cm}p{1cm}p{1cm}p{1cm}p{1cm}p{1cm}}
\\[-1.8ex] \hline 
\hline
 <20 & 20-30 & 30-40 & 40-50& 50-60 & 60-70 & 70-80 & 80-90 & >90 \\ 
 \hline \\[-1.8ex] 
55 & 133 & 222 & 294 & 514 & 562 & 397 & 179 & 26 \\
\hline 
\hline \\[-1.8ex]
 \end{tabular}
\end{table}

\begin{table}[H] \centering 
  \caption{Breakdown Patients by BMI Group} 
\begin{tabular}{p{1cm}p{1cm}p{1cm}p{1cm}p{1cm}p{1cm}p{1cm}p{1cm}p{1cm}p{1cm}p{1cm}}
\\[-1.8ex] \hline 
\hline
 <15 & 15-20 & 20-25 & 25-30 & 30-35 & 35-40 & 40-45 & 45-50 & 50-55 & 55-60 & >60 \\ 
 \hline \\[-1.8ex] 
14 & 158 & 486 & 595 & 467 & 276 & 171 & 106 & 56 & 21 & 32 \\
\hline 
\hline \\[-1.8ex]
 \end{tabular}
\end{table}

Figure 3 shows the average radiation and contrast administered for each
of the age groups, and Figure 4 shows the average radiation and contrast
administered for the BMI groups.

\includegraphics{Updates_Writeup_files/figure-latex/unnamed-chunk-4-1.pdf}

\includegraphics{Updates_Writeup_files/figure-latex/unnamed-chunk-5-1.pdf}

For our age groups, we see a nearly quadratic relationship with
effective dose, with more radiation and contrast being administered for
groups between 30 and 60 years of age, and lower amounts administered to
both younger and older groups. There doesn't appear to be any prominent
relationship between age and contrast.

Now looking at BMI, there appears to be a linear association between
radiation and BMI, with an increase in effective dose as BMI increases.
Again, there does not appear to be any relationship between contrast and
BMI

From these results, it appears that the amount of radiation and contrast
administered to a patient are correlated. Indeed, we find a Pearson's
correlation of 0.47, which is significant (t(2380) = 26.05, p
\textless{} .001).

\subsection{Effective Dose Model}\label{effective-dose-model}

With the fixed values for effective dose, we refit our linear model.
Again, a log-transformation was performed to normalize the effective
dose variable. The initial predictors considered were CT Type, BMI
(\textless{}25, 25-40, \textgreater{}40), Age (\textless{}18, 18-35,
\textgreater{}35), Gender, and Location of Admission. From the initial
model, the only significant location factor was the ICU. In this case,
we condensed the \textit{Location} variable into two groups: patients
admitted in the ICU and other patients. No significant difference in
predictive accuracy was determined between the model with the full
location variable and the model with the condensed location variable, so
we prefer the condensed model. The final model with coefficients is
given in Table 4. The baseline factors for each variable are: BMI:
\textless{}25, Location: ICU, CT: 128SSwIR, Age: \textless{}18, Gender:
Female.

\begin{table}[H] \centering 
  \caption{Effective Dosage Model Coefficients} 
\begin{tabular}{p{3cm}|p{3cm}p{3cm}p{3cm}}
\\[-1.8ex] \hline 
\hline
    & Estimate & Std. Error & p  \\
\hline \\[-1.8ex] 
 Intercept   & 0.472 & 0.111 & .027 $***$ \\
 BMI 25-40   & 0.516 & 0.023 & <.001 $***$\\
 BMI $>$40   & 1.156 & 0.032 & <.001 $***$\\
 Loc: Other  & $-$0.106 & 0.032 & <.001 $***$\\
 CT: 64SSwIR & 0.019 & .049 & .699  \\
 CT: 64SSWoIR & 0.582 & 0.065 & <.001 $***$ \\
 CT: DS Scanner &$-$0.203 & 0.021 & <.001 $***$ \\
 Age: 18-35 & 0.611 & 0.111 & <.001 $***$\\
 Age: $>$35 & 0.624 & 0.107 & <.001 $***$ \\
 Gender: Male & 0.227 & 0.020 & <.001 $***$ \\
\hline 
\hline \\[-1.8ex]
\multicolumn{4}{l}{\textit{Note:} Significance: $.$ < .1, $*$ <.05, $**$ < .01, $***$ < .001}
 \end{tabular}
\end{table}

\paragraph{Comparison of Scanner Type}\label{comparison-of-scanner-type}

Post-hoc analysis using Tukey's HSD was conducted for the difference in
effective dosage administered based on the type of scanner being used.
This analysis found that DS scanners were associated with significantly
lower radiation levels than all of the other scanners (\(p < .001\) for
all 3). Moreover, the 64SSwIR scanner had significantly higher radiation
levels than both the 64SSwoIR and 128SSwIR (\(p<.001\) for both).
Lastly, there was no difference in effective dose between the 64SSwIR
and the 128SSwIR scanners.

The following illustrate the differences in effective dose by scanner as
determined by our model:

\begin{itemize}
\item The DS scanners was associated with 81.6\% of the radiation given for the 128SSwIR scanner and 80.1\% of the radiation used for the 64SSwIR scanner, decreases of 18.4\% and 19.9\% respectively
\item The DS scanners was associated with 45.6\% of the radiation given for the 64SSwoIR scanner, a decrease of 54.4\%
\item The 64SSwoIR scanner was associate with a 67.5\% increase in radiation over the 128SSwIR scanner, and a 64.4\% increase in radiation over the 64SSwIR scanner 
\end{itemize}

To note, since there was no significant difference between the 64SSwIR
and 128SSwIR scanners, the first and third bullet points above can be
generalized. This would tells us that DS scanners use
\textasciitilde{}19\% less radiation than the IR scanners, and the
64SSwoIR scanners use \textasciitilde{}65\% more radiation than the IR
scanners.

\subsection{Contrast Model}\label{contrast-model}

We also refit the contrast model using the updated data set. We
considered a full model including CT Type, BMI (\textless{}25, 25-40,
\textgreater{}40), Age (\textless{}18, 18-35, \textgreater{}35), Gender,
and Location of Admission. Model coefficients are given in Table 5.
Interestingly, with the updated data we no longer have significance for
Age groups, BMI groups, or Gender. This would suggest that some of the
erratic values from the previous dataset that have been removed were
strongly linked to specific age, BMI, and gender levels.

\begin{table}[H] \centering 
  \caption{Contrast Model Coefficients} 
\begin{tabular}{p{3cm}|p{3cm}p{3cm}p{3cm}}
\\[-1.8ex] \hline 
\hline
    & Estimate & Std. Error & p  \\
\hline \\[-1.8ex] 
 Intercept   & 88.580 & 0.592 & .027 $***$ \\
 CT: 64SSwIR & $-$6.953 & 1.831 & <.001 $***$  \\
 CT: 64SSWoIR & $-$0.214 & 2.376 & .928 \\
 CT: DS Scanner &$-$1.799 & 0.751 & .017 $*$ \\
 Loc: ICU & 3.411 & 1.161 & .003 $**$\\
 Loc: IN & 1.635 & 0.900 & .069 $.$ \\
 Loc: OUT & 4.188 & 1.435 & .004 $**$ \\
\hline 
\hline \\[-1.8ex]
\multicolumn{4}{l}{\textit{Note:} Significance: $.$ < .1, $*$ <.05, $**$ < .01, $***$ < .001}
 \end{tabular}
\end{table}

\paragraph{Comparison of Scanner
Type}\label{comparison-of-scanner-type-1}

As with the effective dose model, post hoc analysis using Tukey's HSD
for the type of scanner. Here we find that the only significant
difference was between the 64SSwIR and 64SSwoIR scanners (\(p = .022\)).
From our model, this difference equates to the 64SSwIR scanners using
6.95 units less contrast than the 64SSwoIR scanners.

The interesting part of this finding is that our model found a
significant effect for DS Scanners over 128SSwIR scanners, but the
post-hoc analysis did not. Tukey's HSD tends to be more conservative,
and so it is more difficult to find significance. This is primarily to
account for the number of pair-wise comparisons being made, whereas
standard linear regression does not need to account for this.


\end{document}
