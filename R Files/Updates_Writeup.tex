\documentclass[]{article}
\usepackage{lmodern}
\usepackage{amssymb,amsmath}
\usepackage{ifxetex,ifluatex}
\usepackage{fixltx2e} % provides \textsubscript
\ifnum 0\ifxetex 1\fi\ifluatex 1\fi=0 % if pdftex
  \usepackage[T1]{fontenc}
  \usepackage[utf8]{inputenc}
\else % if luatex or xelatex
  \ifxetex
    \usepackage{mathspec}
  \else
    \usepackage{fontspec}
  \fi
  \defaultfontfeatures{Ligatures=TeX,Scale=MatchLowercase}
\fi
% use upquote if available, for straight quotes in verbatim environments
\IfFileExists{upquote.sty}{\usepackage{upquote}}{}
% use microtype if available
\IfFileExists{microtype.sty}{%
\usepackage{microtype}
\UseMicrotypeSet[protrusion]{basicmath} % disable protrusion for tt fonts
}{}
\usepackage[margin=1in]{geometry}
\usepackage{hyperref}
\hypersetup{unicode=true,
            pdftitle={Update\_Writeup},
            pdfauthor={Aidan Mullan},
            pdfborder={0 0 0},
            breaklinks=true}
\urlstyle{same}  % don't use monospace font for urls
\usepackage{graphicx,grffile}
\makeatletter
\def\maxwidth{\ifdim\Gin@nat@width>\linewidth\linewidth\else\Gin@nat@width\fi}
\def\maxheight{\ifdim\Gin@nat@height>\textheight\textheight\else\Gin@nat@height\fi}
\makeatother
% Scale images if necessary, so that they will not overflow the page
% margins by default, and it is still possible to overwrite the defaults
% using explicit options in \includegraphics[width, height, ...]{}
\setkeys{Gin}{width=\maxwidth,height=\maxheight,keepaspectratio}
\IfFileExists{parskip.sty}{%
\usepackage{parskip}
}{% else
\setlength{\parindent}{0pt}
\setlength{\parskip}{6pt plus 2pt minus 1pt}
}
\setlength{\emergencystretch}{3em}  % prevent overfull lines
\providecommand{\tightlist}{%
  \setlength{\itemsep}{0pt}\setlength{\parskip}{0pt}}
\setcounter{secnumdepth}{0}
% Redefines (sub)paragraphs to behave more like sections
\ifx\paragraph\undefined\else
\let\oldparagraph\paragraph
\renewcommand{\paragraph}[1]{\oldparagraph{#1}\mbox{}}
\fi
\ifx\subparagraph\undefined\else
\let\oldsubparagraph\subparagraph
\renewcommand{\subparagraph}[1]{\oldsubparagraph{#1}\mbox{}}
\fi

%%% Use protect on footnotes to avoid problems with footnotes in titles
\let\rmarkdownfootnote\footnote%
\def\footnote{\protect\rmarkdownfootnote}

%%% Change title format to be more compact
\usepackage{titling}

% Create subtitle command for use in maketitle
\newcommand{\subtitle}[1]{
  \posttitle{
    \begin{center}\large#1\end{center}
    }
}

\setlength{\droptitle}{-2em}
  \title{Update\_Writeup}
  \pretitle{\vspace{\droptitle}\centering\huge}
  \posttitle{\par}
  \author{Aidan Mullan}
  \preauthor{\centering\large\emph}
  \postauthor{\par}
  \predate{\centering\large\emph}
  \postdate{\par}
  \date{2/26/2019}

\usepackage{float}

\begin{document}
\maketitle

\subsection{Effective Dose and
Contrast}\label{effective-dose-and-contrast}

\paragraph{Summary Statistics}\label{summary-statistics}

Numerical summaries of both effective dose and contrast are given in
Table 1. Using the fixed data for effective dosage of radiation
administered to patients, we retrieve the plots shown in Figure 1. Here
we can see the right-skew in the histogram suggesting several patients
recieved much higher amounts of radiation than average.

\begin{table}[H] \centering 
  \caption{Summary Statistics of Radiation and Contrast} 
\begin{tabular}{p{3cm}p{1cm}p{1cm}p{1cm}p{1cm}p{1cm}p{1cm}}
\\[-1.8ex] \hline 
\hline
 & Min & 1st Q & Median & Mean & 3rd Q & Max \\ 
\hline \\[-1.8ex] 
Effective Dose & 0.10 & 3.06 & 4.30 & 5.631 & 6.36 & 48.83 \\
Contrast       & 5.0 & 75.0 & 95.0 & 88.5 & 98.0 & 150.0 \\
\hline 
\hline \\[-1.8ex]
 \end{tabular}
\end{table}

\includegraphics{Updates_Writeup_files/figure-latex/unnamed-chunk-2-1.pdf}

We also have constructed an updated histogram and boxplot for the amount
of contrast given to patients, which are shown in Figure 2. Here we see
a centralized distribution with some extreme values on both sides.

\includegraphics{Updates_Writeup_files/figure-latex/unnamed-chunk-3-1.pdf}

We are also interested in the relationship of the effecive dose and
contrast administered with both the age and BMI of the patient. Patients
were divided into 10-year age groups and 5-unit BMI groups. In both
cases, cut-off values were determined based on the number of patients
that fell into the extreme groups. For example, there was only one
patient in the data under the age of 10, so they were added into the
``\textless{}20'' group. The number of patients that belong to each age
group and BMI group is given in Table 2 and 3 respectively.

\begin{table}[H] \centering 
  \caption{Breakdown Patients by Age Group} 
\begin{tabular}{p{1cm}p{1cm}p{1cm}p{1cm}p{1cm}p{1cm}p{1cm}p{1cm}p{1cm}}
\\[-1.8ex] \hline 
\hline
 <20 & 20-30 & 30-40 & 40-50& 50-60 & 60-70 & 70-80 & 80-90 & >90 \\ 
 \hline \\[-1.8ex] 
55 & 133 & 222 & 294 & 514 & 562 & 397 & 179 & 26 \\
\hline 
\hline \\[-1.8ex]
 \end{tabular}
\end{table}

\begin{table}[H] \centering 
  \caption{Breakdown Patients by BMI Group} 
\begin{tabular}{p{1cm}p{1cm}p{1cm}p{1cm}p{1cm}p{1cm}p{1cm}p{1cm}p{1cm}p{1cm}p{1cm}}
\\[-1.8ex] \hline 
\hline
 <15 & 15-20 & 20-25 & 25-30 & 30-35 & 35-40 & 40-45 & 45-50 & 50-55 & 55-60 & >60 \\ 
 \hline \\[-1.8ex] 
14 & 158 & 486 & 595 & 467 & 276 & 171 & 106 & 56 & 21 & 32 \\
\hline 
\hline \\[-1.8ex]
 \end{tabular}
\end{table}

Figure 3 shows the average radiation and contrast administered for each
of the age groups, and Figure 4 shows the average radiation and contrast
administered for the BMI groups.

\includegraphics{Updates_Writeup_files/figure-latex/unnamed-chunk-4-1.pdf}

\includegraphics{Updates_Writeup_files/figure-latex/unnamed-chunk-5-1.pdf}

Interestingly, we see nearly identical patterns for both the radiation
and contrast administered when divided into either age groups or BMI
groups. For age, we see a nearly quadratic relationship, with more
radiation and contrast being administered for groups between 30 and 60
years of age, and lower amounts administered to both younger and older
groups.

On the other hand, we have seem to have a purely increasing relationship
for radiation and contrast with BMI. As a patient's BMI increases, we
can see that the average effective dosage and amount of contrast given
both increase.

From these results, it appears that the amount of radiation and contrast
administered to a patient are correlated. Indeed, we find a Pearson's
correlation of 0.47, which is significant (t(2380) = 26.05, p
\textless{} .001).


\end{document}
